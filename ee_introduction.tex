%!TEX root= ./master.tex

\chapter{Role of the Electrical Subteam}
The Electrical and Electronic section of MUR Motorsports is a recent addition to the team's personnel. Replacing the Mechatronics section in 2011 due to university changes, the Electrical Subteam is tasked primarily with the responsibility of Power Distribution and Data Acquisition (DAQ). The way this power distribution and data acquisition is physically implemented on the car is in the form of a Wiring 'Loom'. The Loom is a collection of wires and connectors which make the relevant connections required to facilitate data acquisition and power distribution. The design and implementation of the Loom is the highest priority deliverable that MUR requires from the Electrical Subteam. This, however, is not all the team decided it would aim to deliver this year. Being a team of engineers determined to innovate through the exploration of new technologies, several other projects have been undertaken which were determined to be beneficial for the MUR team by both the Electrical team themselves as well as the Integration subteam\footnote{The Integration subteam are responsible for determining key decisions which effect the progress of the project as a whole.}. The projects which have been the main focuses of this year's efforts are detailed below.

\section{Power Distribution}
Of all of the roles of the EE team, power distribution is by far the most critical to the functionality of the car. Without power, the car is unable to start at all. It is for this reason that this year's EE team ensured that a high priority was given to all work towards ensuring adequate power would be delivered to and from all relevant regions of the car via the Loom.

The purpose of the loom first and foremost is to provide sufficient power from the vehicle's electrical generation system to all of the electrical and electronic components of the vehicle, such as the starter motor, ignition coils, ECU and dash data logger.  

The loom must also be correctly packaged in the vehicle, whilst still offering sufficient power rating to provide the required energy to each component. A neat and sufficiently protected loom improves reliability and safety. It is also important to plan the physical design of the loom well so that there is no unnecessary wiring as weight minimisation is imperative. The neatness of the loom is a focal point for judges at the final competition, and thus it is important to deliver a neat finished product.


%\subsection{MOSFET Regulator and Rectifier - maybe remove this}
%The regulator and rectifier unit is a part of the vehicle's electrical generation system and is situated after the alternator.  The two functions work as follows: the rectifier takes the three phase AC signal generated by the alternator and turns it into a rippling ~20-35V DC signal, the regulator operated by sensing what the rectifier does and temporarily shorts one of the three phases until the output voltage is no greater than 14.4V.
%
%The current unit uses thyristors to perform the switching of the phases to ground, and a majority of the heat produced is not from the short circuit but from the switching. This is because the switching is relatively very slow and inaccurate.  It should also be noted that while the regulator is generating much of the heat, the rectifier is also subjected to that heat, being housed in the same unit.  Over a long lifetime this heat degrades the circuits and can cause a number of failures due to the ever-increasing production of heat.  
%
%One way to significantly reduce this effect is to utilise MOSFET transistors to switch instead of thyristors, as they are much more exact and faster to switch causing less heat production and possibly better overall performance.

\section{Data Acquisition and Analysis}
In 2012 a set of Motec equipment designed to assist in the collection of data was purchased. This equipment was an M400 - an Engine Control Unit or ECU, and an ADL3 - Advanced Dash Logger 3. These two devices greatly assist the team in the collection of data from the sensors, since they allow the user to monitor the output of the sensors at all times and record these signals for the duration of a race. However, this is only possible if the loom, either directly or indirectly by virtue of implementation of a CAN bus, connects all sensors to one of these devices. It is important that all sensors are mounted correctly and packaged appropriately to allow for easy connection to the loom, to align with the reliability requirements of the project. Packaging and placement of all electrical and electronic hardware in the vehicle should be determined early in the design phase, working with the other sections and placing the designs in the team's 3D CAD of the vehicle.

\subsection{Sensor Design, Amplification and Calibration, Strain Gauges}
This year there has been considerable effort put into ensuring that the signals being read by the electrical systems put in place by the electrical team correspond to accurate real world units for use by the other subteams in the analysis of their components. Ensuring members of other Subteams are provided with the actual values requested ensures all other subteams can make the most out of the data collected during the extensive physical testing done throughout the year. 

The largest undertaking regarding sensor design was the development of a method for both the application of strain gauges to push/pull rods as well as the amplification of their signal. This was done to ensure accurate values can be provided to other subteams within MUR. This data is critical to the suspension team in the analysis of the system they have implemented so that future years can scrutinise their design decisions.

Similarly to strain gauges, the circuitry to measure the heat in various areas of the car via thermocouples has undergone amplification and calibration so that the temperatures are able to be monitored with confidence. This is important because without being able to rely on the accuracy of the temperatures being provided to the other teams, there will be uncertainty surrounding the design decisions the other subteams base on temperatures provided to them.

\subsection{Live Data Transmission/Telemetry}
By implementing a telemetry system through which all onboard systems can be monitored wirelessly, access to real time data has been provided to the team. In previous years, data was only available for download after the race was completed. With access to the sensor data during a race, new techniques for driver training have become possible, increasing the accuracy with which suggestions can be made to drivers to improve their performance as they are actually driving. This has led to a more prepared group of drivers for the final competition, having a direct correlation with an increase in points, which is in essence the main driving force behind all decisions based around the construction of components on the car.

%\subsection{Track Lap Beacon}
%Whilst GPS logging enabled the team to map the car's position around the track, a much more accurate way to determine lap times was introduced this year, with actual lap beacons. An infra-red beacon sits on the side of the track, which an infra-red receiver in the car detects, signaling a lap has been completed to the ADL3. This is invaluable for driver selection and training and enables the team to get a better idea of how they will perform at the competition.
%
%An off-the-shelf beacon kit was considered, however it was deemed to be excessively expensive for what it is. The shelf bought version allows the user to set up multiple beacons around the track to get section times, as well as the total lap time. Using the information on how our logging equipment interfaces with these beacon kits, we were able to reverse engineer an equivalent system, offering all the same features, for a fraction of the cost.

%\subsection{Traction and Launch Control}
%One of the most simple yet beneficial additions that can be made to the vehicle without any additional hardware is traction and launch control.  Making use of the hall effect sensors currently installed on each of the vehicles wheels, it is possible to retard ignition of the engine should wheel slip be detected. This is a simple implementation of traction control and allows the driver to maximise power transfer from the tyres to the road surface with little effort, directly contributing to competition points.

\section{CAN Bus Automation and the Carduino}
The CAN bus is a protocol which has been adopted universally in the automotive industry for its robustness and ability to effectively provide communications between all components found in automobiles. It is for this reason that this technology has been researched extensively and a device has been constructed to allow future years to hopefully build systems which communicate via the CAN bus, presenting an entire new set of options for automation on board the car.

In order to allow our team to create devices which interface via the CAN protocol, a circuit has been developed which is able to act as a node on the CAN bus. The circuit is called the Carduino. It provides the ability to process instructions according to either sensor inputs plugged directly into the Carduino itself, or via instructions sent over the CAN bus from other areas of the car attached to the bus via either another Carduino or a MOTEC device such as the ADL3 or the M400. It has been designed to use the Arduino IDE since ease of use was prioritised amongst the design objectives. This circuit will allow future teams to develop nodes on the car controlled by the CAN bus to provide actuators, advanced sensors or any other device they require.

%might put this bit in if troy wants to write a section on the wiki
% \section{Knowledge Transferral}

% One of the major challenges faced as each year's team is replaced by the subsequent year's capstone project is the transferral of all the lessons learnt over the course of the year. Without this, each team has to start learning virtually from scratch, making all tasks considerably more difficult. It is for this reason that this year's EE subteam has emphasized the importance of streamlining this knowledge transferral. By building up a knowledge bank in the form of an online wiki, the development of the wiring loom and all other tasks undertaken are made more efficient and time effective.



