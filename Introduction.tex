\chapter{Introduction}

\section{Project Overview}
This report will present the work of the Electrical and Electronic section of the 2013 Melbourne University Racing Motorsports team.  The aim of this project is to design, build and test the electrical and electronic systems for the MUR 2013 race vehicle in preparation for the 2013 Formula SAE-A competition.

\subsection{Formula SAE Background}
Formula SAE is an international competition for student members of the Society of Automotive Engineers for the purpose of designing, building and competing in a open wheel race car, with the intention of being marketed to the amateur weekend autocross driver.

The competition is much more than just a final dynamic race, with the overall scoring depending on the results of 4 dynamic race events (skid pan, acceleration, auto cross and endurance) and 3 static events (engineering design, presentation and cost analysis).  In addition to this competing students are committed to work on a complex and meaningful engineering project in a team environment, and to be successful must utilise skills such as time management, budgeting, fund raising, sourcing materials and components, manufacturing and testing.

The Australiasia event, Formula SAE-A, has been an annual competition since 2000.

\subsection{MUR Core Design Goals}
In 2013 MUR Motorsports have collectively defined three overreaching goals for the team to strive for. The first is that the vehicle must be simple in design and implementation, which aids with keeping costs low, leading to easier and more timely maintenance and repair. The second is that the vehicle must be reliable, so careful consideration is given when pushing the limits of design and the potential downtime this may cause. Significant testing of all components both by themselves, and as part of the vehicle as a whole, is given a high priority. The actual performance of the car is carefully monitored, and all aspects, including driver performance, are assessed to see where improvements to reliability can be made. The third goal is to design the car to ensure it performs well in FSAE events. This requires the car to perform at the upper limit for speed and handling achievable without contradicting the first two goals.

The Electrical Engineering Subteam play a crucial role in the execution of these three design goals. For the first goal of simplicity, all electronics which provide core functionality to the operation of the car have been designed so that all wires are logically and comprehensively labelled, and there is no unnecessary or convoluted routing of wires on the car. This allows all team members working on the electrical components to easily understand what they are dealing with, and increases productivity. This emphasis on simplicity and comprehensive documentation of the electronics also helps to achieve the second goal of reliability, as it allows for easy maintenance when things do go wrong. It also helps to ensure errors are not made in the first place. Finally, in order to achieve the best possible performance out of the car as per the third goal, several additional tasks have been explored this year. Tasks peripheral to the core functionality of the electrical components of the car have been undertaken, all resulting in increases in the performance of either the car or the driver, whilst avoiding compromising either the simplicity or reliability strived for by MUR.
