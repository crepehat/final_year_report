%!TEX root= ./master.tex
\chapter{Traction Control and Launch Control}

\section{Abstract}

\section{Introduction}
Acceleration, as well as being major factor in determining a racing car’s time around a track, is frequently used as a key indicator of the performance of both road cars and track cars. The acceleration test is one of the major challenges in Formula SAE competition, the performance of a team in this test is determined by both driver and vehicle ability. 
There are a number of electrical design modifications that can be made to cars to help drivers accelerate faster and apply power with more confidence when exiting corners, two of these driving aids are Traction Control and Launch Control. 
Launch Control Systems use a car’s Electronic Control Unit to hold the engine at a pre-determined speed prior to take-off, this allows the driver to focus solely on releasing the clutch to engage the car's engine to its gearbox. Removing this variable from racing starts allows engineers to test and set the optimal engine speed for launch and makes for more consistently quick launches of the racing car. 
Traction Control Systems uses a car’s ECU to limit the engine power output upon detection of wheel-spin. Once a car’s wheels begin to spin or slide, the car becomes much more difficult to control and not accelerate as quickly due to the nature of static and sliding friction. For an inexperienced driver, wheel-spin when exiting a corner can be quite daunting and difficult to control. Traction Control allows for use of the maximum amount of power that can be delivered before the rear wheels begin to spin, allowing a driver to put their foot down with significantly more confidence. This makes the car more forgiving for inexperienced drivers not used to the vehicle's power and allows more experienced drivers to easily keep the car at its limits on the race track.
Implementation of these systems on th MUR 2013 car can will be through the use of the MoTeC M400 Advanced Features package which includes TCS and LCS software. The use of these systems will require the installation of wheel-speed sensors, experimental determination of optimal engine launch speed and setup of the MoTeC software.

\subsection{Problem Definition}

\section{Literature Review}
\subsection{Introduction}
Developed primarily through Formula 1 racing in the early 1990's, TCS and LCS have since become prevelent in both the motorsports and automotive industry. The benefits of the driver aids were so prevelent in Forumla 1 that they were banned from competition within months of being introduced (1). To understand the various methods involved with these systems and assess their use on other Formula SAE cars, a range of literature on the technology was reviewed prior to development of the systems.
 
\subsection{Review}
\subsubsection{Traction Control}
Traction Control is defined as "a method of preventing wheels from spinning when traction is applied by limiting the amount of power supplied to the wheel" (2). Traction Control Systems can be broken down into two main wheel control methods, these include: Anti-Lock Braking Systems which also provide barking improvements (3) and Engine Power limiting (4). Rapidly limiting engine power to prevent wheel-slip can also be controlled by a range of methods, not limited to spark reduction/retard (5), fuel reduction (6) and throttle control (4).

\subsection{Conclusion}

\section{Design and Development}

\section{Implementation}


\section{Conclusion}
